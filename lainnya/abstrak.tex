\begin{center}
  \large
  \textbf{PENGEMBANGAN \emph{ANEMOMETER ULTRASONIC} DALAM RUANGAN
  	BERBASIS \emph{RASPBERRY PI PICO}}
\end{center}
\addcontentsline{toc}{chapter}{ABSTRAK}
% Menyembunyikan nomor halaman
\thispagestyle{empty}

\begin{flushleft}
  \setlength{\tabcolsep}{0pt}
  \bfseries
  \begin{tabular}{ll@{\hspace{6pt}}l}
  Nama Mahasiswa / NRP&:& Ghiyas Ash-Shidiqie Rismawan / 07211940000014\\
  Departemen&:& Teknik Komputer FTEIC - ITS\\
  Dosen Pembimbing&:& 1. Dion Hayu Fandiantoro, S.T., M.Eng.\\
  & & 2. Arief Kurniawan, S.T, M.T.\\
  \end{tabular}
  \vspace{4ex}
\end{flushleft}
\textbf{Abstrak}

% Isi Abstrak
Pada ruang bangunan rumah sakit, penghawaan ruang bangunan adalah aliran udara segar didalam ruang bangunan yang memadai untuk menjamin Kesehatan penghuni ruangan (Kepmenkes No.1204/ Menkes/ SK/ X/ 2004).
Aliran udara dalam ruang tersebut dapat berasal dari ventilasi alamiah (seperti jendela) ataupun ventilasi buatan/mekanik (seperti AC atau exhaust). 
Ruangan dengan volume 100 $m^3$ sekurang-kurangnya memiliki 1 fan dengan diameter 50 cm dengan debit udara 0,5 $m^3/detik$ atau 2,55 $m/detik$.
Salah satu jenis velocity anemometer yang dapat digunakan untuk mengukur kecepatan angin adalah ultrasonic anemometer. 
Anemometer ultrasonik memanfaatkan prinsip time of flight ultrasonik untuk memastikan kecepatan dan arah angin.
Penelitian sebelumnya menyebutkan akurasi data pengukuran kecepatan angin yang diambil dari prototype memiliki error hingga 11.11 persen. Suatu perangkat pengukur kecepatan aliran udara idealnya memiliki error kurang dari 6 persen (Jakevicus, 2005). 
Oleh karena itu, dilakukan penelitian ini guna meningkatkan akurasi data serta mampu membaca kecepatan angin minimal 2,55 $m/detik$.

\vspace{2ex}
\noindent
\textbf{Kata Kunci: \emph{Kecepatan Angin, Raspberry Pi Pico, Sensor Ultrasonic}}