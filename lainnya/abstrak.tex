\begin{center}
  \large
  \textbf{PENGEMBANGAN \emph{ANEMOMETER ULTRASONIC} DALAM RUANGAN
  	BERBASIS \emph{RASPBERRY PI PICO}}
\end{center}
\addcontentsline{toc}{chapter}{ABSTRAK}
% Menyembunyikan nomor halaman
\thispagestyle{empty}

\begin{flushleft}
  \setlength{\tabcolsep}{0pt}
  \bfseries
  \begin{tabular}{ll@{\hspace{6pt}}l}
  Nama Mahasiswa / NRP&:& Ghiyas Ash-Shidiqie Rismawan / 07211940000014\\
  Departemen&:& Teknik Komputer FTEIC - ITS\\
  Dosen Pembimbing&:& 1. Dion Hayu Fandiantoro, S.T., M.Eng.\\
  & & 2. Arief Kurniawan, S.T, M.T.\\
  \end{tabular}
  \vspace{4ex}
\end{flushleft}
\textbf{Abstrak}

% Isi Abstrak
Abstrak harus berisi seratus hingga dua ratus kata. \lipsum[1]

\vspace{2ex}
\noindent
\textbf{Kata Kunci: \emph{Roket, Anti-gravitasi, Meong}}