\begin{center}
  \large
  \textbf{DEVELOPMENT OF INDOOR \emph{ULTRASONIC ANEMOMETER} BASED ON \emph{RASPBERRY PI PICO}}
\end{center}
% Menyembunyikan nomor halaman
\thispagestyle{empty}

\begin{flushleft}
  \setlength{\tabcolsep}{0pt}
  \bfseries
  \begin{tabular}{lc@{\hspace{6pt}}l}
  Student Name / NRP&: &Ghiyas Ash-Shidiqie Rismawan / 07211940000014\\
  Department&: &Computer Engineering FTEIC - ITS\\
  Advisor&: &1. Dion Hayu Fandiantoro, S.T., M.Eng.\\
  & & 2. Arief Kurniawan, S.T, M.T.\\
  \end{tabular}
  \vspace{4ex}
\end{flushleft}
\textbf{Abstract}

% Isi Abstrak
The abstract must consist between two hundred to three hundred words. \lipsum[1]

\vspace{2ex}
\noindent
\textbf{Keywords: \emph{Ultrasonic Sensor, Raspberry Pi Pico, Wind Speed}}