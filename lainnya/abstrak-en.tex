\begin{center}
  \large
  \textbf{DEVELOPMENT OF INDOOR \emph{ULTRASONIC ANEMOMETER} BASED ON \emph{RASPBERRY PI PICO}}
\end{center}
% Menyembunyikan nomor halaman
\thispagestyle{empty}

\begin{flushleft}
  \setlength{\tabcolsep}{0pt}
  \bfseries
  \begin{tabular}{lc@{\hspace{6pt}}l}
  Student Name / NRP&: &Ghiyas Ash-Shidiqie Rismawan / 07211940000014\\
  Department&: &Computer Engineering FTEIC - ITS\\
  Advisor&: &1. Dion Hayu Fandiantoro, S.T., M.Eng.\\
  & & 2. Arief Kurniawan, S.T, M.T.\\
  \end{tabular}
  \vspace{4ex}
\end{flushleft}
\textbf{Abstract}

% Isi Abstrak
In hospital building rooms, ventilation of building spaces is a flow of fresh air within the building space that is sufficient to ensure the health of the occupants of the room (Kepmenkes No.1204/Menkes/SK/X/2004).
Airflow in the room can come from natural ventilation (such as windows) or artificial/mechanical ventilation (such as air conditioning or exhaust).
A room with a volume of 100 $m^3$ has at least 1 fan with a diameter of 50 cm with an air flow rate of 0.5 $m^3/sec$ or 2.55 $m/sec$.
One type of velocity anemometer that can be used to measure wind speed is an ultrasonic anemometer.
The ultrasonic anemometer utilizes the ultrasonic time of flight principle to determine wind speed and direction.
Previous research stated that the accuracy of wind speed measurement data taken from the prototype has an error of up to 11.11 percent. 
An air flow velocity measuring device should ideally have an error of less than 6 percent (Jakevicus, 2005).

\vspace{2ex}
\noindent
\textbf{Keywords: \emph{Raspberry Pi Pico, Ultrasonic Sensor, Wind Speed}}