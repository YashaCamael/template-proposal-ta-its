\section{PENDAHULUAN}

\subsection{Latar Belakang}

% Ubah paragraf-paragraf berikut sesuai dengan latar belakang dari tugas akhir
Ruangan adalah suatu tempat tertutup dengan langit-langit yang berada di rumah atau bentuk bangunan lainnya. Ruangan biasanya memiliki pintu dan jendela yang berfungsi sebagai tempat masuknya cahaya, 
aliran udara, dan akses menuju ruangan tersebut. Aliran udara berperan dalam memberikan penghawaan serta mengeluarkan udara dari dalam ke luar ruangan. Namun, aliran udara didalam ruangan tidaklah konstan karena dipengaruhi 
oleh luasnya akses udara, keberadaan suhu dan kelembapan di dalam ruangan.

Untuk mengukur kecepatan aliran udara dapat menggunakan alat bernama Anemometer. Anemometer memiliki berbagai macam jenis, secara umum anemometer terbagi menjadi dua macam yaitu velocity anemometer atau anemometer yang 
berfungsi mengukur kecepatan angin dan anemometer tekanan atau pressure anemometer yang berfungsi mengukur tekanan angin. Salah satu yang merupakan jenis velocity anemometer yaitu sonic/ultrasonic anemometer, dengan menggunakan 
media berupa gelombang suara.

Pada penelitian sebelumnya, telah dibuat suatu rancang bangun anemometer yang digunakan didalam ruangan. Pada rancangan Anemometer Ultrasonic sendiri memiliki kekurangan pada akurasi data karena masih memiliki error yang cukup 
tinggi hingga 11.11 persen. Untuk meningkatkan akurasi data diperlukan frekuensi yang lebih tinggi pada mikrokontroler yang digunakan. Pada mikrokontroler sebelumnya hanya sebatas 16MHz, diharapkan dengan 
mengganti mikrokontroler menjadi Raspberry Pi Pico yang memiliki kemampuan frekuensi hingga 133MHz dapat meningkatkan akurasi data dari anemometer ultrasonic didalam ruangan.

\subsection{Rumusan Masalah}

% Ubah paragraf berikut sesuai dengan rumusan masalah dari tugas akhir
Berdasarkan hal yang telah dipaparkan di latar belakang. Rumusan masalah yang dibawa adalah sebagai berikut:
\begin{enumerate}
	\item[a.] Pada penelitian sebelumnya, terdapat kekurangan pada akurasi data yang didapatkan masih rendah.
	\item[b.] Ukuran dari rancang bangun Anemometer masih terbilang cukup besar untuk digunakan didalam ruangan.
\end{enumerate}

\subsection{Batasan Masalah atau Ruang Lingkup}

Batasan masalah pada Tugas Akhir ini meliputi:
\begin{enumerate}
	\item Alat yang dirancangkan menggunakan anemometer hotwire untuk kalibrasi alat.
	\item Alat yang dirancang menggunakan sensor suhu untuk mengecek temperatur sekitar saat dilakukan kalibrasi alat.
\end{enumerate}

\subsection{Tujuan}

% Ubah paragraf berikut sesuai dengan tujuan penelitian dari tugas akhir
Tujuan dari Tugas Akhir ini adalah sebagai berikut:
\begin{enumerate}
	\item[a.]	Meningkatkan akurasi data yang didapatkan.
	\item[b.]	Memperkecil desain alat dari penelitian diteliti sebelumnya.
\end{enumerate}

\subsection{Manfaat}

% Ubah paragraf berikut sesuai dengan tujuan penelitian dari tugas akhir
Adapun manfaat yang diperoleh dari Tugas Akhir ini sebagai berikut:
\begin{enumerate}
	\item [a.]	Bagi penulis 
	
	Memperluas ilmu dalam perancangan anemometer ultrasonic.
	
	\item [b.]	Bagi institusi
	
	 Memberikan inspirasi dan implementasi yang bermanfaat dalam perancangan anemometer ultrasonic
	
	\item [c.]	Bagi masyarakat
	
	Memperluas wawasan dalam perancangan anemometer ultrasonic dalam pengukuran aliran udara.
	
\end{enumerate}
