\section{METODOLOGI}

% Ubah konten-konten berikut sesuai dengan isi dari metodologi

\subsection{Metode yang digunakan}

\begin{figure}[h!]
	\centering
	\includegraphics[width=0.5\linewidth]{"gambar/Flowchart penelitian.drawio"}
	\caption{Flowchart Penelitian}
	\label{fig:flowchart-penelitian}
\end{figure}

Pada Tugas Akhir ini, garis besar alur Tugas Akhir dapat dilihat pada Gambar \ref*{fig:flowchart-penelitian}. Berikut penjelasan
dari tiap tahapan pada Gambar \ref{fig:flowchart-penelitian}.

\begin{enumerate}
  \item Studi Literatur
  
  Studi literatur adalah metode awal yang dilakukan guna mencari materi-materi penunjang dari pengembangan anemometer ultrasonic 
  dalam ruangan berbasis Raspberry Pi Pico. Studi literatur ini dilakukan dengan mempelajari dan menganalisis dari rancang bangun 
  anemometer yang telah dibuat sebelumnya. Baik berupa penggunaan sensor, aktuator, maupun metode penelitian sebagai bahan penunjang 
  tambahan dari gagasan yang akan direalisasikan. Dalam pencarian studi literatur ini dilakukan referensi dari buku, jurnal penelitian, 
  ataupun artikel ilmiah yang berkaitan dengan gagasan yang dibawa. Data yang sudah terkumpul dari studi literatur sebelumnya dikumpulkan
  dan diolah untuk penunjang gagasan yang direalisasikan.

  \item Persiapan Rancang Bangun Anemometer Ultrasonic

  \begin{figure}[h!]
    \centering
    \includegraphics[width=0.4\linewidth]{"gambar/Blok diagram anemometer ultrasonic"}
    \caption{Blok Diagram Sistem}
    \label{fig:blok-diagram-anemometer-ultrasonic}
  \end{figure}

  Model sistem yang digunakan dapat dilihat pada  gambar \ref{fig:blok-diagram-anemometer-ultrasonic}. Prototipe anemometer ultrasonic dalam ruangan berbasis 
  Raspberry Pi Pico ini akan dilengkapi dengan Website yang dapat menampilkan hasil data yang diperoleh alat secara real-time. Pengujian prototipe ini akan dilakukan 
  dengan menghitung kecepatan angin pada saat kondisi normal dengan jarak konstan, kemudian hasilnya akan dibandingkan dengan anemometer hot-wire komersial. 

  \item Perancangan Perangkat komersial
  
  Pada penelitian kali ini akan dilakukan pembuatan desain alat menggunakan aplikasi Fusion 360, dimana hasilnya akan dicetak menggunakan printer 3D. Selain desain alat, terdapat komponen
  penting lainnya seperti sensor Ultrasonic, sensor suhu, penggaris, dan LCD untuk menampilkan informasi. Setelah seluruh komponen dijadikan prototipe, maka akan dilanjutkan ke tahap kalibrasi

  \item Kalibrasi Sensor Ultrasonic

\begin{figure}[h!]
	\centering
	\includegraphics[width=0.7\linewidth]{"gambar/Diagram blok pengukuran kecepatan angin.drawio"}
	\caption{Blok Diagram Perhitungan Kecepatan Angin}
	\label{fig:diagram-blok-pengukuran-kecepatan-angin}
\end{figure}

  Untuk menemukan konversi dari pembacaan sensor yang tepat. Dilakukan beberapa percobaan dengan berbagai kondisi dan variabel untuk menentukan rumus yang tepat untuk digunakan sebagai rumus umum.
  Adapun diagram blok sistem pengukuran kecepatan angin pada alat ini dapat dilihat pada Gambar \ref{fig:diagram-blok-pengukuran-kecepatan-angin}.

  \item Perancangan Perangkat Lunak
  
  Pada penelitian kali ini dibutuhkan perancangan perangkat lunak untuk menghubungkan antara hasil pembacaan sensor ultrasonic dengan website melalui jaringan internet.

  \item Pembuatan Website untuk menampilkan data alat
  
  Secara garis besar, website akan menampilkan kecepatan angin dari prototipe alat secara real-time.
  
  \item Koneksi Anemometer ultrasonic dengan Website
  
  Pada tahapan ini, akan dilakukan penghubungan antara perangkat keras dengan perangkat lunak sehingga prototipe alat dapat terhubung dengan internet dan data yang diperoleh dapat dilihat melalui 
  website. Jika semua sudah terkoneksi dengan benar maka akan dilanjutkan pada tahap pengujian.

  \item Pengujian alat dan pengambilan data
  
  Pada tahap pengujian akan dilakukan pengujian pengukuran kecepatan angin. Pengujian alat ini menggunakan metode perubahan variabel kecepatan angin kemudian dilakukan perbandingan 
  antara hasil pengukuran dari alat prototipe dengan hasil pengukuran dari anemometer Hot-wire komersial.
  \item Penarikan kesimpulan
  
  Berdasarkan data yang diperoleh dari tahap pengujian, dilakukan analisa dan didapatkan kesimpulan yang menjawab dari perumusan masalah yang terdapat pada bab I. 
\end{enumerate}

\subsection{Bahan dan peralatan yang digunakan}

Untuk Alat dan bahan penunjang penelitian Tugas Akhir ini adalah sebagai berikut:
\begin{enumerate}
  \item Raspberry Pi Pico W
  
  Raspberry Pi Pico W adalah sebuah board minimum sistem mikrokontroller yang berbasis chip RP2040 didalamnya. 
  Alat ini digunakan untuk membaca data dari sensor ultrasonic dan memproses data hasil bacaan sensor. Selain itu, 
  mikrokontroller ini sudah memiliki modul wifi didalamnya sehingga dapat mengirimkan data langsung ke server database 
  melalui jaringan wifi internet.

  \item Modul Ultrasonic US-100
  
  Merupakan sensor ultrasonic yang umumnya digunakan sebagai pengukur jarak, namun pada penelitian ini digunakan sebagia
  pengukur kecepatan angin. Modul sensor ini terdiri dari sepasang transducer ultrasonic, salah satu bagian berfungsi sebagai 
  transmitter yang mengubah sinyal elektrik menjadi sinyal pulsa gelombang suara ultrasonic dengan frekuensi 40KHz., dan bagian lainnya 
  berfungsi sebagai receiver yang berfungsi menerima sinyal gelombang ultrasonic.
  \item Sensor Suhu
  
  Sensor suhu digunakan untuk mengetahui kondisi temperatur sekitar dari alat prototipe yang dibangun. Hal ini berguna sebagai variabel 
  pendukung dikarenakan salah satu parameter dalam kecepatan angin adalah suhu udara sekitar.
\end{enumerate}

\subsection{Urutan pelaksanaan penelitian}

Berikut jadwal pelaksanaan kegiatan Tugas Akhir selama 16 minggu, dapat dilihat pada tabel \ref{tbl:timeline}.
% Ubah tabel berikut sesuai dengan isi dari rencana kerja
\newcommand{\w}{}
\newcommand{\G}{\cellcolor{gray}}
\begin{table}[h!]
  \begin{tabular}{|p{3.5cm}|c|c|c|c|c|c|c|c|c|c|c|c|c|c|c|c|}

    \hline
    \multirow{2}{*}{Kegiatan} & \multicolumn{16}{|c|}{Minggu} \\
    \cline{2-17} &
    1 & 2 & 3 & 4 & 5 & 6 & 7 & 8 & 9 & 10 & 11 & 12 & 13 & 14 & 15 & 16 \\
    \hline

    % Gunakan \G untuk mengisi sel dan \w untuk mengosongkan sel
    Studi Literatur &
    \G & \w & \w & \w & \w & \w & \w & \w & \w & \w & \w & \w & \w & \w & \w & \w \\
    \hline

    Persiapan Alat dan Bahan &
    \w & \G & \G & \w & \w & \w & \w & \w & \w & \w & \w & \w & \w & \w & \w & \w \\
    \hline

    Perancangan Perangkat Keras &
    \w & \w & \w & \G & \G & \w & \w & \w & \w & \w & \w & \w & \w & \w & \w & \w \\
    \hline

    Kalibrasi Sensor &
    \w & \w & \w & \w & \w & \G & \G & \w & \w & \w & \w & \w & \w & \w & \w & \w \\
    \hline

    Perancangan Perangkat Lunak &
    \w & \w & \w & \w & \w & \w & \w & \G & \G & \w & \w & \w & \w & \w & \w & \w \\
    \hline

    Pembuatan Website &
    \w & \w & \w & \w & \w & \w & \w & \w & \w & \G & \G & \w & \w & \w & \w & \w \\
    \hline

    Koneksi Antara Alat dengan Website &
    \w & \w & \w & \w & \w & \w & \w & \w & \w & \w & \w & \G & \w & \w & \w & \w \\
    \hline

    Pengujian Alat &
    \w & \w & \w & \w & \w & \w & \w & \w & \w & \w & \w & \w & \G & \G & \w & \w \\
    \hline

    Penarikan Kesimpulan &
    \w & \w & \w & \w & \w & \w & \w & \w & \w & \w & \w & \w & \w & \w & \G & \G \\
    \hline

  \end{tabular}
  \captionof{table}{Tabel timeline}
  \label{tbl:timeline}
\end{table}

