\section{METODOLOGI}

% Ubah konten-konten berikut sesuai dengan isi dari metodologi

\subsection{Metode yang digunakan}

\begin{figure}[h!]
	\centering
	\includegraphics[width=0.7\linewidth]{"gambar/Flowchart penelitian.drawio"}
	\caption{Flowchart Penelitian}
	\label{fig:flowchart-penelitian}
\end{figure}

Pada Tugas Akhir ini, garis besar alur Tugas Akhir dapat dilihat pada Gambar \ref*{fig:flowchart-penelitian}.
Skema sistem yang akan digunakan pada rencana Tugas Akhir ini ditunjukkan pada Gambar \ref*{fig:blok-diagram-anemometer-ultrasonic}. 
Pengujian performansi alat ini menggunakan metode pengujian manipulasi kecepatan angin kemudian dilakukan perbandingan 
antara sensor Ultrasonic pada Anemometer Ultrasonic dalam Ruangan Berbasis Raspberry Pi Pico dengan anemometer Hot-wire komersial yang
 memiliki pengukuran kecepatan angin paling kecil mencapai 0,1 m/s.

\begin{figure}[h!]
	\centering
	\includegraphics[width=0.7\linewidth]{"gambar/Blok diagram anemometer ultrasonic"}
	\caption{Blok Diagram Sistem}
	\label{fig:blok-diagram-anemometer-ultrasonic}
\end{figure}

Adapun diagram blok sistem pengukuran kecepatan angin pada alat ini adalah sebagai berikut:

\begin{figure}[h!]
	\centering
	\includegraphics[width=0.7\linewidth]{"gambar/Diagram blok pengukuran kecepatan angin.drawio"}
	\caption{Blok Diagram Perhitungan Kecepatan Angin}
	\label{fig:diagram-blok-pengukuran-kecepatan-angin}
\end{figure}


\subsection{Bahan dan peralatan yang digunakan}

Untuk Alat dan bahan yang diperlukan dalam Tuagas Akhir ini adalah sebagai berikut:
\begin{enumerate}
  \item Raspberry Pi Pico W
  \item Sensor Ultrasonic US-100
  \item Sensor Suhu
  \item Anemometer Hot-Wire sebagai alat validasi pengukuran
\end{enumerate}

\subsection{Urutan pelaksanaan penelitian}

% Ubah tabel berikut sesuai dengan isi dari rencana kerja
\newcommand{\w}{}
\newcommand{\G}{\cellcolor{gray}}
\begin{table}[h!]
  \begin{tabular}{|p{3.5cm}|c|c|c|c|c|c|c|c|c|c|c|c|c|c|c|c|}

    \hline
    \multirow{2}{*}{Kegiatan} & \multicolumn{16}{|c|}{Minggu} \\
    \cline{2-17} &
    1 & 2 & 3 & 4 & 5 & 6 & 7 & 8 & 9 & 10 & 11 & 12 & 13 & 14 & 15 & 16 \\
    \hline

    % Gunakan \G untuk mengisi sel dan \w untuk mengosongkan sel
    Perancangan Sistem &
    \G & \G & \G & \G & \G & \G & \w & \w & \w & \w & \w & \w & \w & \w & \w & \w \\
    \hline

    Pengolahan data &
    \w & \w & \w & \w & \w & \w & \G & \G & \G & \w & \w & \w & \w & \w & \w & \w \\
    \hline

    Analisa data &
    \w & \w & \w & \w & \w & \w & \w & \w & \w & \G & \G & \G & \w & \w & \w & \w \\
    \hline

    Evaluasi penelitian &
    \w & \w & \w & \w & \w & \w & \w & \w & \w & \w & \w & \w & \G & \G & \G & \G \\
    \hline

  \end{tabular}
  \captionof{table}{Tabel timeline}
  \label{tbl:timeline}
\end{table}

