\section{METODOLOGI}

% Ubah konten-konten berikut sesuai dengan isi dari metodologi

\subsection{Metode yang digunakan}

\lipsum[11]

% Contoh input gambar dengan format *.jpg
\begin{figure} [ht] \centering
  % Nama dari file gambar yang diinputkan
  \includegraphics[scale=0.45]{gambar/blueprint.jpg}
  % Keterangan gambar yang diinputkan
  \caption{\emph{Blueprint} roket yang akan diuji coba \parencite{SpaceXBlueprint}}
  % Label referensi dari gambar yang diinputkan
  \label{fig:Blueprint}
\end{figure}

% Contoh penggunaan referensi dari gambar yang diinputkan
Pada \emph{blueprint} yang tertera di Gambar \ref{fig:Blueprint}. \lipsum[12]

\subsection{Bahan dan peralatan yang digunakan}

\lipsum[13]
\lipsum[3]

\subsection{Urutan pelaksanaan penelitian}

% Ubah tabel berikut sesuai dengan isi dari rencana kerja
\newcommand{\w}{}
\newcommand{\G}{\cellcolor{gray}}
\begin{table}[h!]
  \begin{tabular}{|p{3.5cm}|c|c|c|c|c|c|c|c|c|c|c|c|c|c|c|c|}

    \hline
    \multirow{2}{*}{Kegiatan} & \multicolumn{16}{|c|}{Minggu} \\
    \cline{2-17} &
    1 & 2 & 3 & 4 & 5 & 6 & 7 & 8 & 9 & 10 & 11 & 12 & 13 & 14 & 15 & 16 \\
    \hline

    % Gunakan \G untuk mengisi sel dan \w untuk mengosongkan sel
    Pengambilan data &
    \G & \G & \G & \G & \w & \w & \w & \w & \w & \w & \w & \w & \w & \w & \w & \w \\
    \hline

    Pengolahan data &
    \w & \w & \w & \w & \G & \G & \G & \G & \w & \w & \w & \w & \w & \w & \w & \w \\
    \hline

    Analisa data &
    \w & \w & \w & \w & \w & \w & \w & \w & \G & \G & \G & \G & \w & \w & \w & \w \\
    \hline

    Evaluasi penelitian &
    \w & \w & \w & \w & \w & \w & \w & \w & \w & \w & \w & \w & \G & \G & \G & \G \\
    \hline

  \end{tabular}
  \captionof{table}{Tabel timeline}
  \label{tbl:timeline}
\end{table}

Pada \emph{timeline} yang tertera di Tabel \ref{tbl:timeline} \lipsum[10]
